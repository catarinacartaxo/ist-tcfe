\section{Simulation Analysis}
\label{sec:simulation}

\subsection{Operating Point Analysis}



\par Comparing to the theoretical analysis results, one notices that the values obtained for each node voltage are the same. However, for the mesh method, $I_1$ and $I_2$ are the same as the ones obtained in the simulation, but $I_3$ is off by 2 orders of magnitude. This huge difference is most likely related to the equations used to solve the system, since the values of $I_1$ and $I_2$ are correct.Consequently, we cannot rule out the precision of the mesh method since in this case the inaccuracy of the values obtained is apparently down to the equations used to achieve them and not the method itself. This means that the Node Voltage Method is an accurate and exact method and can be confidently used to solve complex and extensive electric circuits. From our observations we can't say with conviction that the Mesh Method is as exact and accurate as the the other method used, due to the reasons stated before. 

\begin{table}[h]
  \centering
  \begin{tabular}{|l|r|}
    \hline    
    {\bf Name} & {\bf Value [A or V]} \\ \hline
    i(hvc) & -4.58297e-05\\ \hline
i(va) & -2.23392e-04\\ \hline
@gib[i] & -2.34034e-04\\ \hline
@id[current] & 1.040394e-03\\ \hline
@r1[i] & 2.233917e-04\\ \hline
@r2[i] & -2.34034e-04\\ \hline
@r3[i] & -1.06426e-05\\ \hline
@r4[i] & -1.21796e-03\\ \hline
@r5[i] & 1.274428e-03\\ \hline
@r6[i] & 9.945644e-04\\ \hline
@r7[i] & 9.945644e-04\\ \hline
v(1) & 1.950638e-01\\ \hline
v(2) & -3.28919e-02\\ \hline
v(3) & -5.01937e-01\\ \hline
v(4) & -4.87868e+00\\ \hline
v(6) & 3.863224e+00\\ \hline
v(7) & -6.93173e+00\\ \hline
v(8) & -7.96852e+00\\ \hline
v(9) & -4.87868e+00\\ \hline

  \end{tabular}
  \label{tab:op}
\end{table}











