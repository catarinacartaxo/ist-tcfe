\section{Theoretical Analysis}  \label{sec:analysis}

In order to create an AC/DC converter that converts $230V$ AC to $12V$ DC current,
the first component we need is a transformer. The transformer converts the $230V$ AC
power source to a $12V$ AC power source. To calculate the turn ratio $n$ of the primary
and secondary coils,
we apply the transformer equation

\begin{equation}
    \frac{N_p}{N_s} = \frac{V_p}{V_s}
    \label{transformador}
\end{equation}

\paragraph{} The next component we need is a diode.
A diode is a semiconductor device that allows current to flow in one direction, but prevents it
from flowing in the opposite direction.
\paragraph{} When current is flowing in reverse biased mode, the diode is
off, but when it flows in forward biased mode it turns on. Therefore, the diode allows us to convert
AC into DC. In order to achieve our goal, we decided to use a full wave bridge rectifier.

\paragraph{} We start off by labelling the diodes and the nodes between them, to better explain how
the circuit transforms AC into DC. At the input, the primary side of the transformer, we have an alternating signal, which can be described by a sine wave.
\paragraph{}Considering the positive half
cycle
of the sine wave, on the secondary side of the transformer, one side  is positive and the other negative.
Since conventional current flows from high potential to low potential, the current flows from  the
positive terminal to the negative terminal. Consequently, current flows from the positive terminal
to point $A$ and then it flows to $D_1$ since it is in forward biased mode. It can't flow from point
$A$ to point $D$, since $D_4$ is in reverse biased mode and it blocks any current going in that
direction, unless the voltage is extremely high.
\paragraph{} Ignoring  everything except the wave bridge
rectifier and the load resistor, as current flows from $A$ to $B$ through $D_1$, it flows through
the load resistor and then to point $D$. From $D$, since it makes no sense for current to go back
to $A$ and make a loop, it goes to $C$ through $D_3$ and from there finally to the  negative terminal.
That's the flow of current during the positive half cycle of this circuit. Considering the negative
half cycle of the circuit, the positive and negative terminals switch places and the analysis done
is pretty similar to the one we made for the positive half cycle.
\paragraph{} In either case, the current is
flowing in one direction through the load resistor, from the top to the bottom, and that is the goal,
to create a DC current. At the output we have a DC voltage that is not constant. Now our goal is to
take this pulsating DC output and convert it into a steady DC output. One way to do that is to
introduce a capacitive filter.
\paragraph{}Consequently, by adding a capacitor parallel to the load resistor
we can reduce the fluctuations in voltage at  the output. So, instead of having a pulsating DC
output, the output will more steady with less fluctuations. These fluctuations in the voltage output
are known as the ripple voltage. By increasing the capacitance of the capacitor, the ripple voltage
decreases.  It will not be perfectly constant, but it will be fairly steady for the most part.
\paragraph{}Another thing we have to take into account is the surge current and that is when the capacitor
is charging. While it is charging, a lot of current will be flowing through the capacitor and wires
for a short period of time. So to reduce the surge current and prevent the diodes from burning, we
add a current limiting resistor $R_s$ in series with the wave bridge rectifier and the capacitor.
The value of $R_s$ has to be so the output voltage is unchanged and remains at 12V

\begin{equation}
    V_{out} = \frac{R_L}{R_L + R_s} (V_B - V_D)
    \label{vout}
\end{equation}