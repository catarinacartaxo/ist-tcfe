\section{Simulation Analysis}
\label{sec:simulation}

Starting with the analysis of the oeprating point OP we obtain the following results.
%comparar tabela OP

\begin{table}[!htb]
    \begin{minipage}{0.5\linewidth}
        \centering
        \caption{Simulation Analysis}
        \begin{tabular}{|l|l|}
            \hline
            {\bf Name} & {\bf Value [A or V]} \\ \hline
            yavg & 1.204410e+01\\ \hline
1/((3+1.8+0.4)*(12-yavg+0.000001+ypp)) & -9.37078e+00\\ \hline
ymax-ymin & 2.357000e-02\\ \hline

        \end{tabular}
        \label{tab:tabela4}
    \end{minipage}%
    \begin{minipage}{0.5\linewidth}
        \centering
        \caption{Theoretical Analysis}
        \begin{tabular}{|l|l|}
            \hline
            {\bf Name} & {\bf Value [A or V]} \\ \hline
            \input{../mat/op1teo}
            \input{../mat/op2teo}
        \end{tabular}
    \end{minipage}
\end{table}

\paragraph{} Moving on to the next step, we obtain the input and output impedances and the gain.
The results are showed below
%comparar tabela GAIN ZI ZO

\begin{table}[!htb]
    \begin{minipage}{.5\linewidth}
        \centering
        \caption{Simulation Analysis}
        \begin{tabular}{|l|l|}
            \hline
            {\bf Name} & {\bf Value [A or V]} \\ \hline
            \input{../sim/acsim_tab}
        \end{tabular}
        \label{tab:tabela4}
    \end{minipage}
    \begin{minipage}{0.5\linewidth}
        \centering
        \caption{Theoretical Analysis}
        \begin{tabular}{|l|l|}
            \hline
            {\bf Name} & {\bf Value [A or V]} \\ \hline
            \input{../mat/inc1teo}
            \input{../mat/inc2teo}
        \end{tabular}
    \end{minipage}
\end{table}

As we can see it is very obvious that the results of the voltages and currents are quite similar. Also
the gain and the impedances are quite similar to what is expected. The results were very good overall.
%comparar graficos GAIN

\begin{figure}[h]
    \centering
    \begin{subfigure}{0.25\textwidth}
        \includegraphics[width=\linewidth, clip]{../sim/totalgain.pdf}
    \end{subfigure}
    \begin{subfigure}{0.25\textwidth}
        \includegraphics[width=\linewidth, clip]{../mat/vovi.pdf}
    \end{subfigure}
    \caption{\small Representation of Vo(f)/Vi(f) in a logarithmic scale. }
\end{figure}

The graphics are very similar to each other so we can conclude that the results are good.
